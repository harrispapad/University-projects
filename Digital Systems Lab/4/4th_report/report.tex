\documentclass[12pt]{article}

%change margins
\usepackage[a4paper, left=2cm, right=2cm, top=2.5cm, bottom=2.5cm]{geometry}
%custom title page
\usepackage{titling} 
%forfigures
\usepackage{graphicx}
\usepackage{subcaption}
%placement accuracy
\usepackage{float}
%use arrays
\usepackage{array}
%for advanced math
\usepackage{amsmath}
%for greek text
\usepackage{fontspec}
\usepackage[greek,english]{babel}
\usepackage{polyglossia}
\setdefaultlanguage{greek}
\setotherlanguage{english}
\setmainfont{Palatino Linotype}
\newfontfamily\greekfonttt{FreeMono} % Define monospace font for Greek script

\title{\Large Εργαστήριο \#4: Μοντελοποίηση και προσομοίωση πλακέτας απεικόνισης σε Logisim Evolution}
\author{\textbf{Ομάδα 10} \\
        Παπαδόπουλος Χαράλαμπος - 03120199\\
        Στρουμπάκου Ειρήνη  - 03121183 \\
}
\date{Απρίλιος 2025}

\begin{document}

    \begin{titlepage}
        \centering
        \vspace{3cm}

        \includegraphics[width=4cm]{figures/emp.png}\\
        \vspace{1.5cm}
        {\fontsize{24pt}{20pt}\selectfont{Εθνικό Μετσόβιο Πολυτεχνείο}}\\[0.3cm]
        {\fontsize{16pt}{18pt}\selectfont Σχολή Ηλεκτρολόγων Μηχανικών \& Μηχανικών Υπολογιστών}\\[0.3cm]
        {\fontsize{16pt}{18pt}\selectfont Εργαστήριο Ψηφιακών Συστημάτων}\\[2cm]

        {\bfseries \thetitle \par}
        \vspace{2cm}
        {\theauthor \par}
        \vfill
        \vspace{4cm}

        {\fontsize{15pt}{18pt}Απρίλιος 2025}
    \end{titlepage}

Στην παρούσα εργασία θα αναλύσουμε την πλακέτα απεικόνισης που μας δόθηκε και θα προσομοιώσουμε τη λειτουργία της σε Logisim Evolution. 
Θα αναλύσουμε τη λειτουργία των κυκλωμάτων που την αποτελούν και θα προσομοιώσουμε τη λειτουργία τους.
\newline
Σε αυτό το σημείο η υλοποίηση είναι ακόμα αρκετά απλή και φαίνεται παρακάτω:

\begin{figure}[H]
    \centering
    \includegraphics[width=0.8\textwidth]{figures/circuit.png}
    \caption{Το κύκλωμα σε Logisim Evolution}
    \label{fig:board}
\end{figure}

Καταρχάς, να αναφέρουμε ότι έχουν γίνει οι εξής παραδοχές:
\begin{itemize}
    \item Στο πραγματικό κύκλωμα το ρολόι της πλακέτας υλοποιείται με έναν ταλαντωτή 555, ο οποίος δεν είναι διαθέσιμος στο Logisim Evolution. 
          Έτσι, έχουμε χρησιμοποιήσει ένα ρολόι που παράγει σήμα 1Hz.
    \item Τα δύο κουμπιά ελέγχου προφανώς υλοποιούνται με διακόπτες και pull down resistors, για λόγους απλότητας, όμως, εμείς τα έχουμε προσομοιώσει με push buttons. \\
          Σημειώνουμε ότι η λειτουργία τους είναι ακριβώς η ίδια, δηλαδή το button είναι μόνιμα σε μια κατάσταση λογικού 0 και μόνο όσο είναι πατημένο τίθεται σε λογικό 1.
    \item Τα LED υλοποιούνται με χρήση φωτοδιόδων, οι οποίες δεν είναι διαθέσιμες στο Logisim Evolution. 
          Έτσι, έχουμε χρησιμοποιήσει τα LED του Logisim, προσθέτοντας και πύλες NOT δεδομένου ότι ο DEC (74155) είναι αρνητικής λογικής.
\end{itemize}

Τα βασικότερα σημεία του κυκλώματος που επιδέχονται ανάλυσης είναι τα εξής:
\begin{itemize}

    \item Όπως και στην πραγματική πλακέτα, έτσι και εδώ, τα κουμπιά που χρησιμοποιούμε είναι non-latching, δηλαδή δεν κρατάνε την κατάσταση τους όταν τα αφήσουμε.
          Οπότε, απαιτείται η χρηση κυκλωμάτων μνήμης, διαφορετικά το κύκλωμα (συγκεκριμένα ο counter στον οποίον επενεργούν τα κουμπιά) θα ήταν μόνιμα σε μια κατάσταση "reset".\\
          Από εδώ και στο εξής όποτε λέμε "κουμπί" θα αναφερόμαστε σε αυτό το latching κουμπί (FF + push button). \\
          Διαφορετικά θα λέμε push button για να αναφερθούμε στο κουμπί που πατάει ο παίκτης.

    \item Η ιδανική προσέγγιση είναι μέσω T-FF, τα οποία όμως δεν είναι διαθέσιμα στο εργαστήριο. 
          Έτσι, χρησιμοποιούμε JK-FF με τις εισόδους J και K βραχυκυκλωμένες, το οποίο ισοδυναμεί με T-FF.

    \item Το πρώτο κουμπί χειρίζεται την είσοδο LOAD του counter και στην πραγματικότητα είναι το PLAY-button μας.
          Δηλαδη, μόλις το πατήσει ο παίκτης, το παιχνίδι ξεκινάει, διαφορετικά το κύκλωμα είναι σε κατάσταση idle.

    \item Η λογική με την οποία λειτουργεί το πρώτο κουμπί είναι η εξής: \\
          Η είσοδος του FF είναι συνδεδεμένη με την γη (λογικό 0), δηλαδή ο μόνος τρόπος να αλλάξει κατάσταση είναι μέσω των SET-RESET.\\
          Το push-button χειρίζεται την SET είσοδο του FF, το οποίο σημαίνει ότι μόνο μπορεί να ξεκινήσει το παιχνίδι, όχι να το τελειώσει. \\
          Όσον αφορά την λειτουργία του RESET, θα υλοποιηθεί σε μελλοντικό εργαστήριο.

    \item Το δεύτερο κουμπί χειρίζεται την είσοδο D\_U του counter και στην πραγματικότητα είναι το κουμπί απόκρουσης με το οποίο παίζει ο παίκτης. \\
          Όταν αυτό πατηθεί, η "μπάλα" του παιχνιδιού αντιστρέφει την κατεύθυνση της.

    \item Η λογική με την οποία λειτουργεί το δεύτερο κουμπί είναι η εξής: \\    
          Η είσοδος του FF είναι συνδεδεμένη στο push-button. Δηλαδή, υπό συνθήκες "ηρεμίας" η είσοδος είναι 0 και η κίνηση της μπάλας διατηρεί την κατεύθυνσή της. \\
          Πατώντας όμως το push-button, το FF εισέρχεται σε "inverting" κατάσταση και αντιστρέφει την κατεύθυνση της μπάλας, το οποίο εμείς θεωρούμε απόκρουση.

\end{itemize}     
     
Παρακάτω φαίνονται οι κυματομορφές από την προσομοίωση του κυκλώματος:

\begin{figure}[H]
    \centering
    \includegraphics[width=1\textwidth]{figures/simulation.png}
    \caption{Κυματομορφές από την προσομοίωση του κυκλώματος}
    \label{fig:waves}
\end{figure}

Καταρχάς να αναφέρουμε πως για λόγους ευκολότερης διαχείρισης θέσαμε το ρολόι σε συχνότητα 0.5Hz. \\
Αρχικά, φαίνεται πως μέχρι να πατηθεί το κουμπί PLAY, το κύκλωμα βρίσκεται σε κατάσταση αδράνειας και μένει σταθερά αναμένο το πρώτο LED. \\
Μόλις πατηθεί το κουμπί εκκίνησης (30us) ξεκινάει η κίνηση της μπάλας από LED σε LED με φορά προς τα "κάτω". \\
Αυτό αλλάζει μόλις πατηθεί το κουμπί απόκρουσης (80us), οπότε η κατεύθυνση της μπάλας αντιστρέφεται και αρχίζει να κινείται προς τα "πάνω". \newline

 Τέλος, να αναφέρουμε και πάλι πως αυτήν την στιγμή έχουμε υλοποιήσει μόνο την λογική της απεικόνισης. \\
 Στο επόμενο εργαστήριο θα προχωρήσουμε στην υλοποίηση των σεναρίων που χάνει ο παίκτης, δηλαδή:
 \begin{itemize}
    \item Αποκρούσεις σε μεσαίες θέσεις
    \item Καθυστερημένες αποκρούσεις
\end{itemize}

Στις περιπτώσεις αυτές, το παιχνίδι θα σταματάει έως ότου να πατηθεί εκ νέου το PLAY-button. \\
\end{document}
